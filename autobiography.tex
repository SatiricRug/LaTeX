\documentclass[12pt, letter]{article}

\usepackage{setspace}
\usepackage[margin=1in]{geometry}
\usepackage{indentfirst}
\usepackage{fancyhdr}

\renewcommand{\ss}[2]{#1\textsuperscript{#2}}

\fancypagestyle{plain} {
	\cfoot{}
	\rfoot{\thepage}
	\renewcommand{\headrulewidth}{0pt}
}

\doublespacing
\pagestyle{plain}

\title{\textbf{Autobiography}}
\author{Peter Schinske}
\date{}

\begin{document}
	\singlespacing
	
	\maketitle
	
	\section*{Family of Origin}
	
	\doublespacing
		
	My parents met at a choir concert in Seattle; a mutual friend introduced them after the concert, and they started dating. My mother grew up in Seattle as the youngest of 6 children, and went to Carleton College in Minnesota, where she majored in English. She then went back to Seattle and went to grad school at the University of Washington for library science. My father, by contrast, grew up with one brother in Sacramento, California. He went to Whitman College in Walla Walla, WA and obtained a bachelor's degree in music before moving to Seattle.
	
	In 1990, my mother ended up moving to Massachusetts for a job at the American Antiquarian Society, a research library that specializes in American texts printed in the \ss{18}{th} and \ss{19}{th} centuries. My father ended up following her there in 1991, and they lived together in Worcester, MA. On August 21, 1992, my parents were married; and on July 21, 1994, my mother gave birth to two twin daughters, named Sophia and Emily. Realizing they could not afford daycare, my parents decided to move back to Seattle to live in my mother's house where she grew up, along with my father and my sisters. They then bought a house in 1996, and we live there still.
	
	\singlespacing
	\section*{Birth to Preschool Years}
	\doublespacing
	
	I was born August \ss{12}{th}, 1998, as the \ss{3}{rd} and youngest child. I went to kindergarten at age 3 at Loyal Heights Community Center for 2 years. I don't remember much of preschool, but I can recall a few moments of me playing there. These are probably my first memories; I have a few more, but I don't know when they happened. My parents tell me, however, that I was a very happy, smiling baby; and that the only times I cried for more than ten minutes were when I got home from the hospital and when I got my first shots.
	
	It was predominantly my mother who took care of me, since my dad worked (and still does work) at a 9--5 job. I do remember going to some sort of daycare once, though I do not think that lasted long. My sisters helped too, of course: I can recall quite a few moments of playing with them and having them take care of me. I also learned to read in this section of my life, developing my linguistic abilities as well as my perceiving personality.
	
	\singlespacing
	\section*{Early School Years (K--\ss{5}{th})}
	\doublespacing
	
	I went to kindergarten at Whittier Elementary, but I tested into the Accelerated Progress Program (APP) and went to Lowell Elementary for first through fifth grade. Coincidentally, this is where my mother went to elementary school as well, for she grew up just a few blocks away from it. Outside of school, my parents became my main caretakers, as my sisters were going into middle and high school. I still had fun with my sisters, but they didn't really have to (or want to) take care of me. My relationship with my parents, of course, stayed the same: they were still kind, loving, and helpful, if a little annoying sometimes.
	
	I grew up a fairly introverted child, and had a hard time making friends. Nevertheless, I liked school and usually had fun doing the work. In spring of 2006 (second grade), I joined the Northwest Boychoir at the suggestion of my father, which has turned out to be a major part of my life. The Northwest Boychoir is regarded as one of the best boychoirs in the country, and the top level of the choir performs regularly for sold-out concerts with the Seattle Symphony at Benaroya Hall. Through boychoir, I learned important values of interpersonal intelligence such as self-discipline, teamwork, and commitment. These lessons have become very useful, and I am glad that they were instilled into me at such a young age.
	
	\singlespacing
	\section*{Adolescence---Middle and High School Years}
	\doublespacing
	
	For middle school, I went to Hamilton International Middle School, and continued in the APP program. I was still very much an introverted person at this point, and had few friends. In addition, I was starting to school less and less as the workload got harder. However, during these years I became much more ``street smart'' due to the nature of middle school. Also, I was at this point in the top level of the Northwest Boychoir, and, while the work was hard, it payed off to sing in front of large audiences in places such as Benaroya Hall. 	
	
	It was at this point that I started being tested for ADHD. This was a long and drawn-out process with the doctor, but he finally diagnosed me with minor ADHD. To a certain degree, this explained some of my previous difficulties. I have always had trouble sitting still; when I was younger, my parents would often tell me ``Look with your eyes,'' meaning don't just touch everything in sight. This also, to some extent, explained my mediocre grades in school. They assigned me medication for it, which helps somewhat in decreasing the symptoms of ADHD.
	
	So I then transitioned to High School, and started here at Ingraham. In addition to starting high school, I also graduated from the Northwest Boychoir to VocalPoint! Seattle, which is a group in the same organization as the Boychoir dedicated to teenagers that have been in the Northwest Boychoir and now have changed voices (i.e., tenors and basses instead of sopranos and altos). In addition, VocalPoint also adds teenage women into the group. In VocalPoint, we still do classical music (usually collaborating with the Boychoir) but we also do biannual shows of music from the 1950s through the 1970s. Because of VocalPoint, I not only grew my sense of musical intelligence, but I became a much more extroverted person, and started making friends in VocalPoint and at school. From VocalPoint, I have become initiative, self-confident, and more content of my lifestyle. While I'm not dating, I have obtained a circle of friends that I hang out with regularly; I attribute this entirely to my experiences at VocalPoint. My relationship with my parents, however, has never changed: they are still kind, loving people that I can look up to.
	
	Unfortunately, at this point my sisters turned 18, and they started going away; when I started high school, they also started college. My older sister, Sophia, is at Western Washington University in Bellingham, WA; the other, Emily, is now forming a startup company with her friend in Olympia. This was a rather big change in my life, for I was always used to them living at home. I still see them a lot, but it isn't really the same anymore.
	
	As for the future, I don't really know what I am going to do next. Due to multiple factors, I did not get around to working on college applications and visits last summer, so my dad pitched me the idea of doing a gap year before I go to college. I will probably do so, and take a year off from academia. I may also get a job, and get the experience of working for a wage. I do still want to go to college, though; I have been thinking about majoring in Computer Science, since programming is a hobby of mine. My overarching goal is to prepare myself for the future so that I can live a happy and successful life later.
		
\end{document}
